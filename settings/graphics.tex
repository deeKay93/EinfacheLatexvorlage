% Graphics and Includes
\usepackage[pdftex]{graphicx}
\graphicspath{{assets/img/}}
\DeclareGraphicsExtensions{.pdf,.jpeg,.png,.jpg}
\usepackage{subcaption}
\usepackage{tikz}
\usetikzlibrary{arrows.meta,bending,automata,shapes}
\usepackage[underline=true,rounded corners=false]{pgf-umlsd}

% \FloatBarrier um zu verhindern, dass ein Image in ein falsches Kaiptel rutscht.
\usepackage[section]{placeins}


% Reduce distance before caption
\usepackage[skip=5pt]{caption}

% DirectoryTree
\usepackage[edges]{forest}

\definecolor{folderbg}{RGB}{124,166,198}
\definecolor{folderborder}{RGB}{110,144,169}
\newlength\Size
\setlength\Size{4pt}
\tikzset{%
  folder/.pic={%
      \filldraw [draw=folderborder, top color=folderbg!50, bottom color=folderbg] (-1.05*\Size,0.2\Size+5pt) rectangle ++(.75*\Size,-0.2\Size-5pt);
      \filldraw [draw=folderborder, top color=folderbg!50, bottom color=folderbg] (-1.15*\Size,-\Size) rectangle (1.15*\Size,\Size);
    },
  file/.pic={%
      \filldraw [draw=folderborder, top color=folderbg!5, bottom color=folderbg!10] (-\Size,.4*\Size+5pt) coordinate (a) |- (\Size,-1.2*\Size) coordinate (b) -- ++(0,1.6*\Size) coordinate (c) -- ++(-5pt,5pt) coordinate (d) -- cycle (d) |- (c) ;
    },
}
\forestset{%
declare autowrapped toks={pic me}{},
pic dir tree/.style={%
for tree={%
    folder,
    font=\ttfamily,
    grow'=0,
  },
before typesetting nodes={%
for tree={%
edge label+/.option={pic me},
},
},
},
pic me set/.code n args=2{%
    \forestset{%
      #1/.style={%
          inner xsep=2\Size,
          pic me={pic {#2}},
        }
    }
  },
pic me set={directory}{folder},
pic me set={file}{file},
}